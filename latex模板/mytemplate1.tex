% 使用xelatex编译

\documentclass[withoutpreface,bwprint]{cumcmthesis} %去掉封面与编号页,电子版提交的时候使用。

% 导入宏包
\usepackage[framemethod=TikZ]{mdframed}
\usepackage{url}   % 支持网页链接
\usepackage{subcaption} % 支持子标题
\usepackage{graphicx}
\usepackage{booktabs}
\usepackage{multirow}
\usepackage{geometry}
    \geometry{a4paper,centering,scale=0.75}
\usepackage{amsmath}
\usepackage{mathtools}
\usepackage{enumitem}
\usepackage[framemethod=TikZ]{mdframed}
\usepackage{url}   % 支持网页链接
\usepackage{subcaption} % 支持子标题
\usepackage{amsmath}
\usepackage{float}


\title{模板(此处改为论文标题)}
\tihao{A} % 选题
\baominghao{5124} % 报名号
\schoolname{上海交通大学} % 大学
\membera{ }
\memberb{ }
\memberc{ }
\supervisor{ }
\yearinput{2023} % 日期:年
\monthinput{xx} % 日期:月
\dayinput{xx} % 日期:日

\begin{document}

\maketitle %标题页

\begin{abstract} % 摘要环境
    本文解决的问题,建立的模型和求解结果

    \textbf{对于问题一:} aa

    \textbf{对于问题二:} bbb

    \textbf{对于问题三:} bbb

\keywords{aaa\quad  bbb\quad   ccc\quad  ddd} % 关键词
\end{abstract}

\section{问题重述}
\subsection{问题背景}

穿越沙漠的游戏规则

\subsection{问题重述}

\quad \textbf{问题一:}

\quad \textbf{问题二:}

\quad \textbf{问题三:}

\section{问题分析}
\subsection{问题一}

\subsection{问题二}

\subsection{问题三}

\section{模型假设}
在这里写一些为了求解做的一般性优化

\section{符号说明及名词定义}
\begin{table}[!htbp]

    \centering
    \begin{tabular}{ccccc}
        \toprule[1.5pt]
        符号 & 符号描述 & 单位 \\
        \midrule[1pt]
        B & 负重上限 & kg \\
        T & 穿越沙漠总时间 & 天 \\
        S & 基础收益 & 元 \\
        \bottomrule[1.5pt]
    \end{tabular}
\end{table}

\section{模型准备}


\section{问题一:基于……的模型}

\section{问题二:基于……的模型}

\section{问题三:基于……的模型}


\section{模型评价与分析}

\newpage
%参考文献
\begin{thebibliography}{9}%宽度9
    \bibitem[1]{liuhaiyang2013latex}
    刘海洋.
    \newblock \LaTeX {}入门\allowbreak[J].
    \newblock 电子工业出版社, 北京, 2013.
    \bibitem[2]{mathematical-modeling}
    全国大学生数学建模竞赛论文格式规范 (2020 年 8 月 25 日修改).
    \bibitem{3} \url{https://www.latexstudio.net}
\end{thebibliography}

\newpage
%附录
\begin{appendices}

    \section{模板所用的宏包}
    \begin{table}[htbp]
        \centering
        \caption{宏包罗列}
        \begin{tabular}{ccccc}
            \toprule
            \multicolumn{5}{c}{模板中已经加载的宏包} \\
            \midrule
            amsbsy & amsfonts & {amsgen} & {amsmath} & {amsopn} \\
            amssymb & amstext & {appendix} & {array} & {atbegshi} \\
            atveryend & auxhook & {bigdelim} & {bigintcalc} & {bigstrut} \\
            bitset & bm    & {booktabs} & {calc} & {caption} \\
            caption3 & CJKfntef & {cprotect} & {ctex} & {ctexhook} \\
            ctexpatch & enumitem & {etexcmds} & {etoolbox} & {everysel} \\
            expl3 & fix-cm & {fontenc} & {fontspec} & {fontspec-xetex} \\
            geometry & gettitlestring & {graphics} & {graphicx} & {hobsub} \\
            hobsub-generic & hobsub-hyperref & {hopatch} & {hxetex} & {hycolor} \\
            hyperref & ifluatex & {ifpdf} & {ifthen} & {ifvtex} \\
            ifxetex & indentfirst & {infwarerr} & {intcalc} & {keyval} \\
            kvdefinekeys & kvoptions & {kvsetkeys} & {l3keys2e} & {letltxmacro} \\
            listings & longtable & {lstmisc} & {ltcaption} & {ltxcmds} \\
            multirow & nameref & {pdfescape} & {pdftexcmds} & {refcount} \\
            rerunfilecheck & stringenc & {suffix} & {titletoc} & {tocloft} \\
            trig  & ulem  & {uniquecounter} & {url} & {xcolor} \\
            xcolor-patch & xeCJK & {xeCJKfntef} & {xeCJK-listings} & {xparse} \\
            xtemplate & zhnumber &       &       &  \\
            \bottomrule
        \end{tabular}%
        \label{tab:addlabel}%
    \end{table}%
    
    以上宏包都已经加载过了,不要重复加载它们。
    
    \section{排队算法--matlab 源程序}
    
    \begin{lstlisting}[language=matlab]
    kk=2;[mdd,ndd]=size(dd);
    while ~isempty(V)
    [tmpd,j]=min(W(i,V));tmpj=V(j);
    for k=2:ndd
    [tmp1,jj]=min(dd(1,k)+W(dd(2,k),V));
    tmp2=V(jj);tt(k-1,:)=[tmp1,tmp2,jj];
    end
    tmp=[tmpd,tmpj,j;tt];[tmp3,tmp4]=min(tmp(:,1));
    if tmp3==tmpd, ss(1:2,kk)=[i;tmp(tmp4,2)];
    else,tmp5=find(ss(:,tmp4)~=0);tmp6=length(tmp5);
    if dd(2,tmp4)==ss(tmp6,tmp4)
    ss(1:tmp6+1,kk)=[ss(tmp5,tmp4);tmp(tmp4,2)];
    else, ss(1:3,kk)=[i;dd(2,tmp4);tmp(tmp4,2)];
    end;end
    dd=[dd,[tmp3;tmp(tmp4,2)]];V(tmp(tmp4,3))=[];
    [mdd,ndd]=size(dd);kk=kk+1;
    end; S=ss; D=dd(1,:);
     \end{lstlisting}
    
     \section{规划解决程序--lingo源代码}
    
    \begin{lstlisting}[language=c]
    kk=2;
    [mdd,ndd]=size(dd);
    while ~isempty(V)
        [tmpd,j]=min(W(i,V));tmpj=V(j);
    for k=2:ndd
        [tmp1,jj]=min(dd(1,k)+W(dd(2,k),V));
        tmp2=V(jj);tt(k-1,:)=[tmp1,tmp2,jj];
    end
        tmp=[tmpd,tmpj,j;tt];[tmp3,tmp4]=min(tmp(:,1));
    if tmp3==tmpd, ss(1:2,kk)=[i;tmp(tmp4,2)];
    else,tmp5=find(ss(:,tmp4)~=0);tmp6=length(tmp5);
    if dd(2,tmp4)==ss(tmp6,tmp4)
        ss(1:tmp6+1,kk)=[ss(tmp5,tmp4);tmp(tmp4,2)];
    else, ss(1:3,kk)=[i;dd(2,tmp4);tmp(tmp4,2)];
    end;
    end
        dd=[dd,[tmp3;tmp(tmp4,2)]];V(tmp(tmp4,3))=[];
        [mdd,ndd]=size(dd);
        kk=kk+1;
    end;
    S=ss;
    D=dd(1,:);
     \end{lstlisting}
    \end{appendices}
    
    
 
 \end{document}



